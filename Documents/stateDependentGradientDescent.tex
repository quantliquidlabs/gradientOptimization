\documentclass{amsproc}
%%%%%%%%%%%%%%%%%%%%%%%%%%%%%%%%%%%%%%%%%%%%%%%%%%%%%%%%%%%%%%%%%%%%%%%%%%%%%%%%%%%%%%%%%%%%%%%%%%%%%%%%%%%%%%%%%%%%%%%%%%%%%%%%%%%%%%%%%%%%%%%%%%%%%%%%%%%%%%%%%%%%%%%%%%%%%%%%%%%%%%%%%%%%%%%%%%%%%%%%%%%%%%%%%%%%%%%%%%%%%%%%%%%%%%%%%%%%%%%%%%%%%%%%%%%%
\usepackage{amsfonts}
\usepackage{amsmath}
\usepackage{enumerate}
\usepackage{graphicx}
%\usepackage{graphix}


\setcounter{MaxMatrixCols}{10}
%TCIDATA{OutputFilter=LATEX.DLL}
%TCIDATA{Version=5.50.0.2953}
%TCIDATA{<META NAME="SaveForMode" CONTENT="1">}
%TCIDATA{BibliographyScheme=Manual}
%TCIDATA{Created=Monday, January 04, 2010 15:01:23}
%TCIDATA{LastRevised=Friday, February 12, 2010 15:39:34}
%TCIDATA{<META NAME="GraphicsSave" CONTENT="32">}
%TCIDATA{<META NAME="DocumentShell" CONTENT="Standard LaTeX\Standard LaTeX Article">}
%TCIDATA{Language=American English}
%TCIDATA{CSTFile=40 LaTeX article.cst}

\newtheorem{theorem}{Theorem}
\newtheorem{acknowledgement}[theorem]{Acknowledgement}
\newtheorem{algorithm}[theorem]{Algorithm}
\newtheorem{axiom}[theorem]{Axiom}
\newtheorem{case}[theorem]{Case}
\newtheorem{claim}[theorem]{Claim}
\newtheorem{conclusion}[theorem]{Conclusion}
\newtheorem{condition}[theorem]{Condition}
\newtheorem{conjecture}[theorem]{Conjecture}
\newtheorem{corollary}[theorem]{Corollary}
\newtheorem{criterion}[theorem]{Criterion}
\newtheorem{definition}[theorem]{Definition}
\newtheorem{example}[theorem]{Example}
\newtheorem{exercise}[theorem]{Exercise}
\newtheorem{lemma}[theorem]{Lemma}
\newtheorem{notation}[theorem]{Notation}
\newtheorem{problem}[theorem]{Problem}
\newtheorem{proposition}[theorem]{Proposition}
\newtheorem{remark}[theorem]{Remark}
\newtheorem{solution}[theorem]{Solution}
\newtheorem{summary}[theorem]{Summary}
%\newenvironment{proof}[1][Proof]{\noindent\textbf{#1.} }{\ \rule{0.5em}{0.5em}}

\newcommand{\s}{\mathbb{S}^1}
\newcommand{\E}{\mathbf{E}}
\newcommand{\Pp}{\mathbf{P}}
\newcommand{\R}{\mathbb{R}}
\newcommand{\N}{\mathbb{N}}
\newcommand{\Z}{\mathbb{Z}}
\newcommand{\cov}{\mathop{\mathsf{cov}}}
\newcommand{\Var}{\mathop{\mathsf{Var}}}
\newcommand{\supp}{\mathop{\mathrm{supp}}}
\newcommand{\ONE}{{\bf 1}}
\newcommand{\eps}{\varepsilon}
\newcommand{\bpf}[1][Proof]{{\noindent {\sc #1: }}}
\newcommand{\epf}{{{\hspace{4 ex} $\Box$ \smallskip}}}
\newcommand{\wb}{\bar}
\newcommand{\x}{X_\eps}
\newcommand{\y}{Y_\eps}
\newcommand{\hatx}{\hat X_\eps}
\newcommand{\tauz}{\tau_0}

\newcommand{\Meps}{\Phi_\eps}
\newcommand{\tauh}{\hat \sigma_\eps}
\newcommand{\taua}{\wb \tau_\eps}
\newcommand{\Fc}{\mathcal{F}}

\title{Scaling Limit for the Diffusion Exit Problem}
\author{Sergio A.Almada}

\begin{document}
\section{Pedro Talk}

The standard gradient descent dynamics is given by
\[
d x (t) = - \nabla U ( x(t) )dt + \sigma dW(t), \quad x(0) = x_0.
\]

In this note we will try to study the modified dynamics
\begin{align} \notag
d\x(t) &=  -e^{ -\gamma \left (  U(\x(t)) - \min_{s\leq t} U(\x(s))  \right ) } \nabla U ( \x(t) )dt + \eps dW(t), \\ 
\x(0)  &= x_0  \label{eqn: master}
\end{align}


\section{Idea of Analysis}
In the analysis of~\eqref{eqn: master}, let us assume that $x_0$ is a local minimum and a stationary point; that is, $\nabla U(x_0) = 0$. Define the deterministic flow induced by the vector field $\nabla U$ starting at an arbitrary point $x\in \R^d$:
\begin{equation} \label{eqn: master_zero_noise}
\frac{d}{dt} S^t x = -\nabla U ( S^t x), \quad S^0 x = x.
\end{equation}
Then, consider the proxy process to~\eqref{eqn: master} given by 
\begin{align} \label{eqn: hat}
d \hatx (t) =  -e^{ -\gamma \left (  U(\hatx(t)) - \min_{s < t - \tau_i} U(S^s \hatx(\tau_i))  \right ) } \nabla U ( \hatx(t) )dt + \eps dW(t), 
\end{align}
on the interval $\tau_i < t < \tau_{i+1}$, where the times $\tau_i$ are the innovation times defined via $$\tau_i = \inf \left\{  t \geq \tau_{i-1}: U( \hatx (t) ) < U( \hatx (\tau_{i-1})  \right \},$$ with $\tau_0 = 0$ and the same initial condition as $\x$, $\hatx (0) = x_0$. Let us now make some observations. 

The time $\tau_1$ is bounded below by the exit time of $\hatx$ from the basin of attraction, $\mathcal{B}(x_0)$, of $x_0$; that is,  
$\tau_1 > \sigma(x_0) = \inf \left \{ t: \hatx(t) \in \partial \mathcal{B}(x_0) \right \}$, since for every $y \in \mathcal{B}(x_0)$, $\nabla U(y)\not = 0$.  Now, from FW theory, it is well known that $\sigma(x_0)$ is exponentially distributed with mean $\eps^{-2} \left( \min_{ y \in \partial \mathcal{B}(x_0) } V(y) - V(x_0) \right )$, where $V$ is the quasi-potential of~\eqref{eqn: hat}. Since, $\hatx$ does not find a new minimum of the function $U$ before time $\sigma(x_0)$, and since $\min_{t>0} U(S^t x_0) = U(x_0)$ we observe that the drift in equation~\eqref{eqn: hat} is given by 
\[
b(x) = - \gamma^{-1} \nabla \left( 1 -  e^{ -\gamma \left (  U(y) - U(x_0)  \right ) } \right),
\]
so the quasi-potential is given by $V(y) = 2 - 2e^{ -\gamma \left (  U(y) - U(x_0)  \right ) }.$
As a consequence, the exit time $\sigma(x_0)$ is exponentially distributed with mean $\lambda$ given by
\begin{align*}
\log \lambda &= \frac{2}{\gamma \eps^2} \left( 1 -  e^{ -\gamma \left (  \min_{ y \in \partial \mathcal{B}(x_0) } U(y) - U(x_0)  \right ) }  \right).
\end{align*}
So that if this expression is of constant order, then for some constant $c>0$ and with $\Delta_{ x_0 } = \min_{ y \in \partial \mathcal{B}(x_0) } U(y) - U(x_0)$ it follows that, for $\gamma$ and $\eps$ small enough,
\begin{align*}
\gamma \Delta_{x_0} &= \log \left( 1 - \gamma \eps^2 c \right) \\
&=  \log\left(  1 + \sum_{i\geq 1} \gamma^i \eps^{2i} c^i \right)\\
&\approx \gamma \eps^2 c + \gamma^2 \eps^4 c^2.
\end{align*}
This quadratic equation has a unique strictly positive solution given by $\eps^4 c^2 \gamma = \Delta_{x_0} - \eps^2 c$, which leads to the following:

\begin{claim}
By choosing $\gamma = \mathcal{O}( \eps^{-4} )$, as $\eps \to 0$, the exit from the basin of attraction $\mathcal{B}(x_0)$ happens in almost constant time.
\end{claim}

\section{Some Potential Discretizations}
Let us consider SDE~\eqref{eqn: master} up to time $\tau_1$, so that the law of $\x$ is the same as the law of 
\[
d\y(t) = - e^{ - \gamma \left(  U(\y(t)) - U(x_0) \right )} \nabla U(\y(t)) dt + \eps dW(t)
\]
Then observe that, if we create a uniform equidistant discretization, $\mathcal{P}=\left\{ 0=t_0 < t_1 <.... \right \}$ of $[0,\infty)$ with $t_1 = \eta$, then it follows that 
\begin{align*}
\y(t_{i+1})  &= \y(t_i) - \int_{t_i}^{t_{i+1}}   e^{ - \gamma \left(  U(\y(s)) - U(x_0) \right )} \nabla U(\y(s))ds + \eps \left( W(t_{i+1}) - W(t_i) \right) \\
&\approx \y(t_i) - \eta \left(   e^{ - \gamma \left(  U(\y(t_i)) - U(x_0) \right )} \nabla U(\y(s))ds + \frac{\eps}{\sqrt{\eta}} \xi_i \right).
\end{align*}
By comparing the last expression with the characterization of the algorithm, it follows that $\eps = \sqrt{\eta}$, so that $\gamma \approx \eta^{-2}$ giving rise to the claim. 
\begin{claim}
Let $y_0 = x_0$, and consider the algorithm,  
\[
y_{i+1} = y_i  - \eta \left( e^{ -\eta^{-2} \left( U_i - U^*_i  \right) } \nabla U_i + \xi_i \right), \quad i \in \mathbb{N},
\]
where we have used the short hand notation $U_i = U(y_i)$ and $U^*_i = \min_{ j \leq i } U_j $. Then, with high probability, after $\mathcal{O}(\gamma^{-1})$ time steps, $y_i$ scapes the set $\mathcal{B}(x_0)$.
\end{claim}
Now, as explained below, exit from the basin of attraction occurs in time

\end{document}